\documentclass[10pt]{article}

\title{Term project report}
\newcommand{\course}{CS 555: Data Science}

\author{Sasha Malone}

\usepackage[xetex]{graphicx} % allow embedded images
  \setkeys{Gin}{width=\linewidth,totalheight=\textheight,keepaspectratio}
  \graphicspath{{graphics/}} % set of paths to search for images
  \usepackage{adjustbox}
\usepackage{amsmath}  % extended mathematics
\usepackage{amssymb}  % extended mathematics
\usepackage{mathtools}
\usepackage{amsthm}   % theorem styles
\usepackage{booktabs} % book-quality tables
\usepackage{units}    % non-stacked fractions and better unit spacing
\usepackage{multicol} % multiple column layout facilities
% \usepackage{fullpage} % reduce margins
\usepackage[margin=1in,tmargin=0.7in,includehead,includefoot,headheight=12pt,heightrounded]{geometry}

\usepackage{hyperref}

\usepackage{lmodern}
\usepackage{xcolor}
\usepackage{listings}
\usepackage{fancyhdr}
\pagestyle{fancy}

\makeatletter
\lhead{\course \\ \vspace{1pt} \large \@title}
\rhead{\textit{\today} \\ \vspace{1pt} \large \@author}
\makeatother

\usepackage{enumerate}

\usepackage{titlesec}
\colorlet{punct}{red!60!black}
\definecolor{background}{HTML}{EEEEEE}
\definecolor{delim}{RGB}{20,105,176}
\colorlet{numb}{magenta!60!black}

\lstdefinelanguage{json}{
    basicstyle=\normalfont\ttfamily,
    showstringspaces=false,
    breaklines=true,
    literate=
     *{:}{{{\color{punct}{:}}}}{1}
      {,}{{{\color{punct}{,}}}}{1}
      {\{}{{{\color{delim}{\{}}}}{1}
      {\}}{{{\color{delim}{\}}}}}{1}
      {[}{{{\color{delim}{[}}}}{1}
      {]}{{{\color{delim}{]}}}}{1},
}


\usepackage{fontspec}
\setmainfont{Century Expanded Regular.otf}[
    Path =  /usr/share/fonts/OTF/,
    BoldFont = Century Expanded Bold.otf,
    ItalicFont = Century Expanded Italic.otf
]
\makeatletter
\renewcommand{\maketitle}{\bgroup\setlength{\parindent}{0pt}
    \begin{center}
    \LARGE \textbf{\@title}
    % \vspace{1pt}
    % \large \textit{\@author \quad $\vert$ \quad \course \quad $\vert$\quad \@date}
    \end{center}
\egroup
}
\makeatother

% Standardize command font styles and environments
\newcommand{\doccmd}[1]{\texttt{\textbackslash#1}}% command name -- adds backslash automatically
\newcommand{\docopt}[1]{\ensuremath{\langle}\textrm{\textit{#1}}\ensuremath{\rangle}}% optional command argument
\newcommand{\docarg}[1]{\textrm{\textit{#1}}}% (required) command argument
\newcommand{\docenv}[1]{\textsf{#1}}% environment name
\newcommand{\docpkg}[1]{\texttt{#1}}% package name
\newcommand{\doccls}[1]{\texttt{#1}}% document class name
\newcommand{\docclsopt}[1]{\texttt{#1}}% document class option name
\newenvironment{docspec}{\begin{quote}\noindent}{\end{quote}}% command specification environment

\newtheorem{thm}{Theorem}
\theoremstyle{definition}
\newtheorem{defn}{Definition}
\newtheorem{exer}{Exercise}
\newtheorem*{exer*}{Exercise}
\theoremstyle{remark}
\newtheorem*{soln}{Solution}
\newtheorem*{remark}{Note}

\newcommand{\ds}{\displaystyle}
\renewcommand{\geq}{\geqslant}
\renewcommand{\leq}{\leqslant}

\providecommand{\tightlist}{%
      \setlength{\itemsep}{0pt}\setlength{\parskip}{0pt}}

\DeclarePairedDelimiter{\abs}{\lvert}{\rvert}
\DeclarePairedDelimiter{\inner}{\langle}{\rangle}
\DeclarePairedDelimiter{\norm}{\lVert}{\rVert}

\begin{document}

\emph{NB:} As the total filesize of my data and project code exceeds
Blackboard's upload limit, please note that it is publicly available on Github
at URL \texttt{http://github.com/sverona/meleeWHR}. Replication instructions
are available in the repository's readme file.

\hypertarget{background-and-problem-description}{%
\section{Background and problem description}\label{background-and-problem-description}}

\emph{Super Smash Bros.~Melee}, a fighting game
originally released in 2001, has recently experienced a resurgence in
competitive play. Major tournament series such as Genesis and Evolution
regularly attract upwards of 1,000 entrants {[}1{]}, a scale otherwise
unheard of for decade-old games incapable of online play.

Melee It On Me (\emph{MIOM}), the competitive community's primary
governing body, produces annual ranking lists of the 100 most skilled
players, established by panel vote. While the general accuracy of these
lists is widely accepted, both the placement of individual players and
the rankings' general methodology are highly debated within the community
at large. There have been many attempts at using existing rating systems such
as Elo and Glicko to generate comparable rankings, but few are still
publicly maintained, partially due to the scattered nature of the
available data. Further, these rankings (like their MIOM counterparts)
are produced at infrequent intervals from aggregated resuls. Thus, since
the scene lacks a formal competitive circuit, the task of ranking the
top players attending an upcoming tournament (e.g., for seeding
purposes) falls mainly on that tournament's organizers.

\hypertarget{objectives}{%
\section{Objectives}\label{objectives}}

The goals of this project were to

\begin{enumerate}
\def\labelenumi{\roman{enumi}.}
\tightlist
\item
  consolidate the available tournament data, dating as far back as
  2003-4, into a publicly maintained dataset;
\item
  use this dataset and a maximum-likelihood estimation method such as
  Whole-History Rating (\emph{WHR}) {[}2{]} to reconstruct real-time
  ratings for the period comprising the earliest MLG tournaments in
  2005-6 to the present day;
\item
  use D3.js {[}3{]} to visualize these time-series in a manner similar
  to {[}4{]}.
\end{enumerate}

\section{Overview of methodology and research question}
The primary question of interest is to derive a stable measure of player
skill from the sequence of match records. One method of doing so that
has been adopted by other e-sports communities is TrueSkill {[}5{]},
which assumes that player skill before any given match follows a
Gaussian distribution. The probability of one player defeating another
is then roughly approximated by the probability that a randomly sampled
value from the former's distribution exceeds one from the latter's; this
gives rise to the \emph{Bradley-Terry model}
\begin{align*}
    P_t(i > j) \approx \frac{\mu_{it}}{\mu_{it} + \mu_{jt}},
\end{align*}
where $\mu_{it}, \mu_{jt}$ are the means of players $i$ and $j$'s
distributions at time $t$. The TrueSkill algorithm, like many others,
performs maximum-likelihood estimation of these parameters using this
assumption. One goal of this project is to adapt this algorithm slightly
to model some of the peculiarities of \emph{SSBM}'s competitive
environment, as demonstrated, e.g., in {[}6{]}.

The WHR algorithm differs from TrueSkill in that it performs MLE over the
player's entire rating history (hence the name.) 

\hypertarget{basic-data-model-and-project-workflow}{%
\section{Data cleaning}\label{basic-data-model-and-project-workflow}}

The data consists primarily of match metadata from tournament brackets,
as shown in the sample below at left. This data was sourced from Liquipedia,
using a scraper written in Python, and cleaned and reformatted into JSON, as
in the sample below at right. It contains the match data from a set between
top-level players at the tournament Shine 2017:

\lstset{frame=single,
        basicstyle=\small\ttfamily,
        basewidth=0.5em,
        breaklines=true,
        postbreak=\mbox{\textcolor{red}{$\hookrightarrow$}\space}
    }
\begin{figure}[!ht]
    \begin{minipage}[t][][t]{.46\textwidth}
        \begin{lstlisting}
|l1m3p1=S2J |l1m3p1flag=us |l1m3p1score=3
|l1m3p2=HugS |l1m3p2flag=us |l1m3p2score=1
|l1m3win=1
|l1m3p1char1=cf |l1m3p2char1=samus |l1m3p1stock1=0 |l1m3p2stock1=1 |l1m3win1=2 |l1m3stage1=Yoshi's Story
|l1m3p1char2=cf |l1m3p2char2=samus |l1m3p1stock2=1 |l1m3p2stock2=0 |l1m3win2=1 |l1m3stage2=Pokémon Stadium
|l1m3p1char3=cf |l1m3p2char3=samus |l1m3p1stock3=2 |l1m3p2stock3=0 |l1m3win3=1 |l1m3stage3=Yoshi's Story
|l1m3p1char4=cf |l1m3p2char4=samus |l1m3p1stock4=3 |l1m3p2stock4=0 |l1m3win4=1 |l1m3stage4=Yoshi's Story
|l1m3date=August 26, 2017
|l1m3details={{BracketMatchDetails|reddit=|comment=|vod=https://www.youtube.com/watch?v=7zTSvNM-E1c}}
        \end{lstlisting}
    \end{minipage} \hfill
    \begin{minipage}[t][][t]{.5\textwidth}
        \begin{lstlisting}[language=json]
"l1m3": {
    "date": "August 26, 2017",
    "details": {
        "comment": "",
        "reddit": "",
        "vod": "https://www.youtube.com/watch?v=7zTSvNM-E1c"
    },
    "p1": "S2J",
    "p1char1": "cf", "p1char2": "cf", "p1char3": "cf", "p1char4": "cf",
    "p1flag": "us",
    "p1score": "3",
    "p1stock1": "0", "p1stock2": "1", "p1stock3": "2", "p1stock4": "3",

    "p2": "HugS",
    "p2char1": "samus", "p2char2": "samus", "p2char3": "samus", "p2char4": "samus",
    "p2flag": "us",
    "p2score": "1",
    "p2stock1": "1", "p2stock2": "0", "p2stock3": "0", "p2stock4": "0",

    "stage1": "Yoshi's Story", "stage2": "Pok\u00e9mon Stadium", "stage3": "Yoshi's Story", "stage4": "Yoshi's Story",

    "win": "1",
    "win1": "2", "win2": "1", "win3": "1", "win4": "1"
}
        \end{lstlisting}
    \end{minipage}
\end{figure}

\section{Results}

\section{Directions for further improvement}

\begin{thebibliography}{7}
\bibitem{size} https://www.ssbwiki.com/List\_of\_largest\_Smash\_tournaments
\bibitem{whr} https://www.remi-coulom.fr/WHR/WHR.pdf 
\bibitem{d3} https://d3js.org/
\bibitem{abacaba} https://www.youtube.com/watch?v=z2DHpW79w0Y
\bibitem{ttt} https://papers.nips.cc/paper/3331-trueskill-through-time-revisiting-the-history-of-chess.pdf
\bibitem{lua} https://www.reddit.com/r/SSBM/comments/4pitia/an\_objective\_ranking\_system\_that\_compensates\_for/
\bibitem{liquid} http://liquipedia.net/smash/Main\_Page
\bibitem{data} http://liquipedia.net/smash/index.php?title=Shine/2017/Melee/Singles\_Bracket\&action=edit\&section=4
\end{thebibliography}
\end{document}
