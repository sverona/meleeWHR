\documentclass[10pt]{article}

\title{Term project proposal}
\newcommand{\course}{CS 555: Data Science}

\author{Sasha Malone}

\usepackage[xetex]{graphicx} % allow embedded images
  \setkeys{Gin}{width=\linewidth,totalheight=\textheight,keepaspectratio}
  \graphicspath{{graphics/}} % set of paths to search for images
  \usepackage{adjustbox}
\usepackage{amsmath}  % extended mathematics
\usepackage{amssymb}  % extended mathematics
\usepackage{mathtools}
\usepackage{amsthm}   % theorem styles
\usepackage{booktabs} % book-quality tables
\usepackage{units}    % non-stacked fractions and better unit spacing
\usepackage{multicol} % multiple column layout facilities
% \usepackage{fullpage} % reduce margins
\usepackage[margin=0.5in,tmargin=0.4in,includehead,includefoot,headheight=12pt,heightrounded]{geometry}

\usepackage{hyperref}

\usepackage{fancyhdr}
\pagestyle{fancy}

\makeatletter
\lhead{\course \\ \vspace{1pt} \large \@title}
\rhead{\textit{\today} \\ \vspace{1pt} \large \@author}
\makeatother

\usepackage{enumerate}
\usepackage{titlesec}

\usepackage{fontspec}
\setmainfont{OldStandard-Regular.ttf}[
    Path =  /usr/share/fonts/TTF/,
    BoldFont = OldStandard-Bold.ttf,
    ItalicFont = OldStandard-Italic.ttf
]
\newfontfamily\titlefont{Baskerville-120-Pro_6076.ttf}[
    Path = /usr/share/fonts/TTF/,
    BoldFont = Baskerville-120-Pro-Bold_6072.ttf,
    ItalicFont = Baskerville-120-Pro-Italic_6073.ttf,
    BoldItalicFont = Baskerville-120-Pro-Bold-Italic_6071.ttf
]

\makeatletter
\renewcommand{\maketitle}{\bgroup\setlength{\parindent}{0pt}
    \begin{center}
    \LARGE \textbf{\@title}
    % \vspace{1pt}
    % \large \textit{\@author \quad $\vert$ \quad \course \quad $\vert$\quad \@date}
    \end{center}
\egroup
}
\makeatother

% Standardize command font styles and environments
\newcommand{\doccmd}[1]{\texttt{\textbackslash#1}}% command name -- adds backslash automatically
\newcommand{\docopt}[1]{\ensuremath{\langle}\textrm{\textit{#1}}\ensuremath{\rangle}}% optional command argument
\newcommand{\docarg}[1]{\textrm{\textit{#1}}}% (required) command argument
\newcommand{\docenv}[1]{\textsf{#1}}% environment name
\newcommand{\docpkg}[1]{\texttt{#1}}% package name
\newcommand{\doccls}[1]{\texttt{#1}}% document class name
\newcommand{\docclsopt}[1]{\texttt{#1}}% document class option name
\newenvironment{docspec}{\begin{quote}\noindent}{\end{quote}}% command specification environment

\newtheorem{thm}{Theorem}
\theoremstyle{definition}
\newtheorem{defn}{Definition}
\newtheorem{exer}{Exercise}
\newtheorem*{exer*}{Exercise}
\theoremstyle{remark}
\newtheorem*{soln}{Solution}
\newtheorem*{remark}{Note}

\newcommand{\ds}{\displaystyle}
\renewcommand{\geq}{\geqslant}
\renewcommand{\leq}{\leqslant}

\providecommand{\tightlist}{%
      \setlength{\itemsep}{0pt}\setlength{\parskip}{0pt}}

\DeclarePairedDelimiter{\abs}{\lvert}{\rvert}
\DeclarePairedDelimiter{\inner}{\langle}{\rangle}
\DeclarePairedDelimiter{\norm}{\lVert}{\rVert}

\begin{document}

\hypertarget{background-and-problem-description}{%
\section{Background and problem
description}\label{background-and-problem-description}}

\emph{Super Smash Bros.~Melee} (hereafter \emph{SSBM},) a fighting game
originally released in 2001, has recently experienced a resurgence in
competitive play. Major tournament series such as Genesis and Evolution
regularly attract upwards of 1,000 entrants {[}1{]}, a scale otherwise
unheard of for decade-old games incapable of online play.

Melee It On Me (\emph{MIOM}), the competitive community's primary
governing body, produces annual ranking lists of the 100 most skilled
players, established by panel vote. The accuracy of these lists and
their methodology are highly debated within the community at large.
There have been many attempts at using existing rating systems such as
Elo and Glicko to generate comparable rankings, but few are still
publicly maintained, partially due to the scattered nature of the
available data. Further, these rankings (like their MIOM counterparts)
are produced at infrequent intervals from aggregated resuls. Thus, since
the scene lacks a formal competitive circuit, the task of ranking the
top players attending an upcoming tournament (e.g., for seeding
purposes) falls mainly on that tournament's organizers.

\hypertarget{objectives}{%
\section{Objectives}\label{objectives}}

I propose to

\begin{enumerate}
\def\labelenumi{\roman{enumi}.}
\tightlist
\item
  consolidate the available tournament data, dating as far back as
  2003-4, into a publicly maintained dataset;
\item
  use this dataset and a maximum-likelihood estimation method such as
  Whole- History Rating (\emph{WHR}) {[}2{]} to reconstruct real-time
  ratings for the period comprising the earliest MLG tournaments in
  2005-6 to the present day;
\item
  use D3.js {[}3{]} to visualize these time-series in a manner similar
  to {[}4{]}.
\end{enumerate}

\hypertarget{basic-data-model-and-project-workflow}{%
\section{Basic data model and project
workflow}\label{basic-data-model-and-project-workflow}}

The data will consist primarily of match metadata from tournament
brackets, as shown in the code snippet below. This data will be cleaned
and reformatted into a format more appropriate for storage in a
relational database, most likely JSON or CSV.

The baseline schema (omitting unnecessary attributes) has the following
tables:
\begin{enumerate}[i.]
    \item TOURNAMENTS, having attribute \texttt{ID} and \texttt{date};
    \item MATCHES, having attributes \texttt{ID}, \texttt{bracket\_ID},
        \texttt{winner\_ID}, \texttt{loser\_ID};
    \item PLAYERS, having attribute \texttt{ID};
    \item MATCHES\_PLAYERS, join table.
\end{enumerate}

At the basic level, within this database, tournaments are ordered
temporally by their \texttt{date} attributes, while their brackets are
ordered temporally by the match attributes \texttt{winner\_ID} and
\texttt{loser\_ID}, which contain the IDs of the matches the winner and
loser of that match play next. \texttt{loser\_ID} is \texttt{NULL} only
if the loser is eliminated from the tournament; \texttt{winner\_ID} is
\texttt{NULL} only if the winner of the match is the winner of the
tournament.

The primary question of interest is to derive a stable measure of player
skill from the sequence of match records. One method of doing so that
has been adopted by other e-sports communities is TrueSkill {[}5{]},
which assumes that player skill before any given match follows a
Gaussian distribution. The probability of one player defeating another
is then roughly approximated by the probability that a randomly sampled
value from the former's distribution exceeds one from the latter's; this
gives rise to the \emph{Bradley-Terry model}
\begin{align*}
    P_t(i > j) \approx \frac{\mu_{it}}{\mu_{it} + \mu_{jt}},
\end{align*}
where $\mu_{it}, \mu_{jt}$ are the means of players $i$ and $j$'s
distributions at time $t$. The TrueSkill algorithm, like many others,
performs maximum-likelihood estimation of these parameters using this
assumption. One goal of this project is to adapt this algorithm slightly
to model some of the peculiarities of \emph{SSBM}'s competitive
environment, as demonstrated, e.g., in {[}6{]}.

\hypertarget{sample-data-format}{%
\section{Sample data format}\label{sample-data-format}}

Currently, the data is primarily stored on sites such as Liquipedia
{[}7{]} in a markup format. The following snippet (sourced from {[}8{]})
contains the match data from three sets between top-level players at the
tournament Shine 2017:

\begin{verbatim}
|l1m1p1=PewPewU |l1m1p1flag=us |l1m1p1score=3
|l1m1p2=Trif |l1m1p2flag=es |l1m1p2score=2
|l1m1win=1
|l1m1p1char1=marth |l1m1p2char1=peach |l1m1p1stock1= |l1m1p2stock1=0 |l1m1win1=1 |l1m1stage1=Unknown
|l1m1p1char2=marth |l1m1p2char2=peach |l1m1p1stock2= |l1m1p2stock2=0 |l1m1win2=1 |l1m1stage2=Unknown
|l1m1p1char3=marth |l1m1p2char3=peach |l1m1p1stock3=0 |l1m1p2stock3=1 |l1m1win3=2 |l1m1stage3=Dream Land
|l1m1p1char4=marth |l1m1p2char4=peach |l1m1p1stock4=0 |l1m1p2stock4=1 |l1m1win4=2 |l1m1stage4=Yoshi's Story
|l1m1p1char5=marth |l1m1p2char5=peach |l1m1p1stock5=1 |l1m1p2stock5=0 |l1m1win5=1 |l1m1stage5=Pokémon Stadium
|l1m1date=August 26, 2017
|l1m1details={{BracketMatchDetails|reddit=|comment=|vod=}}

|l1m2p1=Westballz |l1m2p1flag=us |l1m2p1score=3
|l1m2p2=Lucky |l1m2p2flag=us |l1m2p2score=0
|l1m2win=1
|l1m2p1char1=falco |l1m2p2char1=fox |l1m2p1stock1=1 |l1m2p2stock1=0 |l1m2win1=1 |l1m2stage1=Battlefield
|l1m2p1char2=falco |l1m2p2char2=fox |l1m2p1stock2=2 |l1m2p2stock2=0 |l1m2win2=1 |l1m2stage2=Dream Land
|l1m2p1char3=falco |l1m2p2char3=fox |l1m2p1stock3=1 |l1m2p2stock3=0 |l1m2win3=1 |l1m2stage3=Pokémon Stadium
|l1m2details={{BracketMatchDetails|vod=https://www.youtube.com/watch?v=4bMa_nRjC3E|comment=}}

|l1m3p1=S2J |l1m3p1flag=us |l1m3p1score=3
|l1m3p2=HugS |l1m3p2flag=us |l1m3p2score=1
|l1m3win=1
|l1m3p1char1=cf |l1m3p2char1=samus |l1m3p1stock1=0 |l1m3p2stock1=1 |l1m3win1=2 |l1m3stage1=Yoshi's Story
|l1m3p1char2=cf |l1m3p2char2=samus |l1m3p1stock2=1 |l1m3p2stock2=0 |l1m3win2=1 |l1m3stage2=Pokémon Stadium
|l1m3p1char3=cf |l1m3p2char3=samus |l1m3p1stock3=2 |l1m3p2stock3=0 |l1m3win3=1 |l1m3stage3=Yoshi's Story
|l1m3p1char4=cf |l1m3p2char4=samus |l1m3p1stock4=3 |l1m3p2stock4=0 |l1m3win4=1 |l1m3stage4=Yoshi's Story
|l1m3date=August 26, 2017
|l1m3details={{BracketMatchDetails|reddit=|comment=|vod=https://www.youtube.com/watch?v=7zTSvNM-E1c}}
\end{verbatim}

\begin{thebibliography}{7}
\bibitem{size} https://www.ssbwiki.com/List\_of\_largest\_Smash\_tournaments
\bibitem{whr} https://www.remi-coulom.fr/WHR/WHR.pdf 
\bibitem{d3} https://d3js.org/
\bibitem{abacaba} https://www.youtube.com/watch?v=z2DHpW79w0Y
\bibitem{ttt} https://papers.nips.cc/paper/3331-trueskill-through-time-revisiting-the-history-of-chess.pdf
\bibitem{lua} https://www.reddit.com/r/SSBM/comments/4pitia/an\_objective\_ranking\_system\_that\_compensates\_for/
\bibitem{liquid} http://liquipedia.net/smash/Main\_Page
\bibitem{data} http://liquipedia.net/smash/index.php?title=Shine/2017/Melee/Singles\_Bracket\&action=edit\&section=4
\end{thebibliography}
\end{document}
